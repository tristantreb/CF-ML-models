\chapter{Introduction}

"If it were not for the great variability between individuals, medicine might as well be a science, not an art", Sir William Oslen, 1892. This statement still applies more than one century later, as clinicians still rely much on their personal and colleagues' experience to best treat patients, forming "mental models" for each disease. However, at the interface of biomedical signal processing, computational modelling, machine learning, and health informatics, computational healthcare turns medicine from art to science. From performing machine learning powered mechanistic studies, to communicating interpretable findings, computational healthcare can enable bespoke medicine, empower healthcare professionals, improve care pathways and public health policy, catalyse new therapeutics, and eventually improve population health.

Patients are complex due to their specific genetic backgrounds, medical histories, lifestyles, and distinct environmental exposure. This translates into different competing risks, variations in symptoms, distinct disease trajectories, and different responses to treatment. To address this, machine learning is promising as it can adapt the modelisation complexity to embrace patient variabilities. A comprehensive patient view can be built to provide interpretable analytics to clinicians. Deploying machine learning models in healthcare however needs to exploit a sufficient amount of data, taking the form of electronic health records, physiological time-series, genomics, and proteomics data. Diseases like Cystic Fibrosis (CF) that require continuous life-long monitoring can benefit the most from machine learning applications. Indeed, CF patients follow trimonthly clinics where many types of data are captured to generate the most adequate picture of the patient's health status \cite{gartner_2019}.

CF is the most common life-limiting genetic disorder in Caucasian populations that affects over 90,000 individuals worldwide \cite{bell_mall_2020}. It is caused by biallelic mutations in the Cystic Fibrosis Transmembrane conductance Regulator (CFTR) gene that, in the lung, lead to reduced airway surface liquid, impaired muco-ciliary clearance, chronic bacterial infections and, consequently, progressive inflammatory lung damage until premature death \cite{ratjen_bell_2015}\cite{elborn_2016}. Once considered as a pediatric disease, the majority of patients are now adults thanks to the remarkable advances in CF care improving life expectancy \cite{cff_registry_2019}. The most recent progress are CFTR modulators therapies that target the protein defect at the origin of the disease, thus greatly improving the airway microbiology \cite{rogers_2020}. Among them, the transformative triple therapy called Trikafta or Kaftrio is progressively being included as a regular treatment in the US since 2019, and in Europe since 2020.

Notwithstanding the promising progress on the path to cure CF, morbidity and mortality continue to be driven by episodes of sudden clinical deterioration, termed acute pulmonary exacerbations (APEs), from which patients do not recover fully. The succession of those episodes cause permanent loss of lung function \cite{sanders_2010}, impaired quality of life \cite{britto_2002}, and eventually premature death \cite{liou_2001}. Although the pathophysiology of, and the triggers for, APEs remain unclear, they are usually associated with: worsening symptoms of cough, breathlessness, and fatigue; increased volumes of purulent sputum; decreases in spirometry and sometimes oxygen saturations \cite{bilton_2011}. The lungs ineffectively recover due to a deficiency in the patients’ inflammatory response. Worse, the related dysregulation of the immune system participates in the patient's chronic inflammation status \cite{cantin_2015}, which requires external intervention including treatment with hospital or domiciliary antibiotics \cite{smith_2011}.

Given the importance of APEs and recovery on survival and wellbeing of individuals with CF, there is a critical need to better characterise the physiological changes preceding an APE and following a treatment. Whilst there has been an effort to characterise APEs, a focus on the recovery from antibiotic treatment is rare. Recent work analysed predictors for the onset of a successful recovery, but have been limited to a single feature (FEV1) and low temporal frequency (more than a month), in 2017 \cite{morgan_2017}\cite{sanders_2017}. They showed that patients do not return to 90\% of their baseline lung function in approximately 25\% of recoveries. However, questions such as "How does a recovery look like?, "Are there different types of recoveries?" remain unanswered. Since 2019, a large amount of multidimensional physiological data has been collected through two UK CF home monitoring studies, SMARTCARE and Project Breathe. They involved hundreds of patients self-reporting a dozen bio-markers on a daily basis \cite{damian}. The SMARTCARE data set proved to be rich enough to 1) characterise the typical profile of an APE, 2) infer different types of APEs, 3) predict their onset, using machine learning methods. Therefore it is wondered whether the study of APE could be complemented by the characterisation of the recovery after an antibiotic treatment, using the same machine learning methods.


% antibiotic resistance
