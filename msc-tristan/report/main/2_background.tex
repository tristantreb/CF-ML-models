\chapter{Background}  \label{sec:background}

\subsubsection{The Project Breathe study}
Project Breathe is an ongoing remote health home monitoring study for Cystic Fibrosis patients. It was created in 2019, by the UK Cystic Fibrosis Trust, the University of Cambridge, Royal Papworth Hospital, Magic Bullet, Microsoft, and Microsoft Research; perpetuating a successful previous 2-year feasibility study called SMARTCARE \cite{bach_2020}.

Patient’s individual data is collected on a daily basis through a free publicly available smartphone app, by using a Fitbit to track calories, sleep, and resting heart rate and weight via connected scales, an oximeter that measures blood oxygen levels, a spirometer that gauges lung function. That data is automatically uploaded to the app, and patients also enter self-reported information on how much they are coughing and how they’re feeling overall, as well as temperature. The complete list of measures is described in Table \ref{tab:measures}. Study data were link-anonymised and uploaded to a secure central server for collation and processing.

Project Breathe has two main purposes. Firstly, to allow patients to undertake a part of their follow-up remotely, through virtual clinics, where the data generated at home is automatically made available for the clinician in the hospital. This has multiple advantages. Patients 1) gain control over their health by also looking at the evolution of their measurements, 2) save time by not going to the hospital in person, and 3) have reduced risk of contracting nosocomial bacterial infections, as well as COVID-19. Hospitals patients' management is lightened and clinics can be programmed when they are really needed. By early 2020, Cardiff Hospital also started utilising Project Breathe. Two additional hospitals in the UK and a multi-site center in Canada are currently in the process of recruiting to Project Breathe.
Secondly, it provides multidimensional longitudinal CF patient data for CF research. Until now, half a million measurements were recorded, which makes it the richest available data set to study CF with such a high temporal frequency.

\subsubsection{Insight into the Project Breathe data at our disposal}
As program collaborators, access was granted to the participants from the Royal Papworth Hospital and Cardiff Hospital. It consists of 258 adults with CF, with characteristics broadly representative of the UK adult CF population \cite{uk_resigtry_2019}: mean age of 31.6 years, an average FEV1 of 69.0\% predicted, and 50\% homozygous for the CFTR F508del mutation (figure \ref{fig:breathestats}).
In addition, each study center was asked to provide clinical metadata for each participant including demographic details, sputum microbiology, and CFTR sequencing results; details and dates of hospital admissions and intravenous and oral antibiotic treatment courses and CFTR modulator therapy; and hospital-based measurement of lung function, weight, and C-reactive protein. Both home monitoring data and clinical metadata were then subjected to rigorous quality control (figure \ref{fig:dataqualitycheck}). Half of the individuals were given at least one antibiotic treatment during their enrollment time (figure \ref{fig:enrolmenttime}). This low amount can be explained by the introduction of effective CFTR modulators that are likely to mitigate the amount of APEs (table \ref{tab:cftrmodulators}), and by the short history for the recently enrolled participants (figure \ref{fig:nintr}). Compared to conventional clinical measurements, that are limited to periods when patients attend hospital (figure \ref{fig:clinic}), home monitoring provides an extremely rich dataset (figure \ref{fig:home}) with clear changes in signals in the period following the start of antibiotic treatment, supporting our attempt to characterise recovery with such data.

\subsubsection{Related work: characterisation of APEs}
Recoveries are preceded by APEs. Depending on the type of the decline (e.g. full decline, partial decline, no decline), the recovery is likely to behave differently. Sutcliffe et al. analysed APEs using data from the SMARTCARE study \cite{damian}. They studied 8 features, namely cough, wellness, FEV1, O2 saturation, activity in number of steps, pulse rate, and sleep from 104 adult patients. The main differences between SMARTCARE and Project Breathe are that the first was 2-year time-bounded with patients from seven centers required to record data for at least 6 months while the second is ongoing since 2019 with one center, extended to three others. With a long-term mindset, participants of Project Breathe were slightly less committed to the recording of measurements but provided more data.

The aim was triple: 1) validate that multimodal physiological data contained a signal that could be used as a proxy for the patient's health status, 2) define the profile of an APEs, and 3) predict the onset of APEs. The first goal was validated by successful results for the others. This work's focus is on point 2): they created a probabilistic inference machine learning model, using a Bayesian approach and with convergence through expectation maximisation, that was able to align APEs profiles on the exacerbation start, the onset of the decline in the different features. This was used to draw the typical profile of an APE and to infer multiple types of exacerbations, where three main classes were found. It was decided to mirror this study by characterising recoveries utilising the same machine learning model with requested adaptation and improvements. The next chapter details the project's aim and objectives.
