\chapter{Aim \& Objectives}
% \addcontentsline{toc}{chapter}{Aims \& Objectives}
\label{sec:obj}

\subsubsection{Main project}
This thesis aims to improve CF care management by the characterisation of recovery after antibiotic treatment. This subject was approached with the following mindset: ask clinical questions to the data, answer them with machine learning and statistical methods, and eventually precise the results through feedback from the clinicians. The 5 objectives were:

\begin{enumerate}
    \item Review the literature of CF recovery after a treatment
    \item Develop an expert eye to extracting recovery information through analysing patients' multivariate longitudinal data
    \item Understand adapt the machine learning model for Bayesian inference with expectation maximisation, eventually update it where necessary and optimise its core parameters
    \item Infer the profile of a typical recovery from antibiotic treatment
    \item Investigate if multiple types of recoveries can be inferred using the algorithm
\end{enumerate}

\subsubsection{Side project}
A side project was performed to estimate the variability in forced expiratory volume in 1s (FEV1) measurements. This was initially requested for other research projects in the Department of Medicine. However, the model could demonstrate the impact, or the absence of impact, of  CFTR modulators therapies, mainly Symkevi and Trikafta on the variability. This was done during the two first months of the thesis as part of the introduction to Project Breathe's data and in parallel with the recovery work. It is described in section \ref{sec:sideproject}. The 4 objectives were:
\begin{enumerate}
    \item Review the literature of FEV1 variability estimation
    \item Build a model to estimate the variability in FEV1 measurements
    \item Review the literature for hypothesis testing methods to test homoscedasticity, i.e. equality of variance.
    \item Use the model to demonstrate the effect of CFTR modulators and distinguish patients with similar lung health
\end{enumerate}